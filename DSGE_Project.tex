\documentclass[11pt,preprint]{elsarticle}

\usepackage{lmodern}
%%%% My spacing
\usepackage{setspace}
\setstretch{1.2}
\DeclareMathSizes{12}{14}{10}{10}

% Wrap around which gives all figures included the [H] command, or places it "here". This can be tedious to code in Rmarkdown.
\usepackage{float}
\let\origfigure\figure
\let\endorigfigure\endfigure
\renewenvironment{figure}[1][2] {
    \expandafter\origfigure\expandafter[H]
} {
    \endorigfigure
}

\let\origtable\table
\let\endorigtable\endtable
\renewenvironment{table}[1][2] {
    \expandafter\origtable\expandafter[H]
} {
    \endorigtable
}


\usepackage{ifxetex,ifluatex}
\usepackage{fixltx2e} % provides \textsubscript
\ifnum 0\ifxetex 1\fi\ifluatex 1\fi=0 % if pdftex
  \usepackage[T1]{fontenc}
  \usepackage[utf8]{inputenc}
\else % if luatex or xelatex
  \ifxetex
    \usepackage{mathspec}
    \usepackage{xltxtra,xunicode}
  \else
    \usepackage{fontspec}
  \fi
  \defaultfontfeatures{Mapping=tex-text,Scale=MatchLowercase}
  \newcommand{\euro}{€}
\fi

\usepackage{amssymb, amsmath, amsthm, amsfonts}

\def\bibsection{\section*{References}} %%% Make "References" appear before bibliography


\usepackage[numbers]{natbib}

\usepackage{longtable}
\usepackage[margin=2.3cm,bottom=2cm,top=2.5cm, includefoot]{geometry}
\usepackage{fancyhdr}
\usepackage[bottom, hang, flushmargin]{footmisc}
\usepackage{graphicx}
\numberwithin{equation}{section}
\numberwithin{figure}{section}
\numberwithin{table}{section}
\setlength{\parindent}{0cm}
\setlength{\parskip}{1.3ex plus 0.5ex minus 0.3ex}
\usepackage{textcomp}
\renewcommand{\headrulewidth}{0.2pt}
\renewcommand{\footrulewidth}{0.3pt}

\usepackage{array}
\newcolumntype{x}[1]{>{\centering\arraybackslash\hspace{0pt}}p{#1}}

%%%%  Remove the "preprint submitted to" part. Don't worry about this either, it just looks better without it:
\makeatletter
\def\ps@pprintTitle{%
  \let\@oddhead\@empty
  \let\@evenhead\@empty
  \let\@oddfoot\@empty
  \let\@evenfoot\@oddfoot
}
\makeatother

 \def\tightlist{} % This allows for subbullets!

\usepackage{hyperref}
\hypersetup{breaklinks=true,
            bookmarks=true,
            colorlinks=true,
            citecolor=blue,
            urlcolor=blue,
            linkcolor=blue,
            pdfborder={0 0 0}}


% The following packages allow huxtable to work:
\usepackage{siunitx}
\usepackage{multirow}
\usepackage{hhline}
\usepackage{calc}
\usepackage{tabularx}
\usepackage{booktabs}
\usepackage{caption}


\newenvironment{columns}[1][]{}{}

\newenvironment{column}[1]{\begin{minipage}{#1}\ignorespaces}{%
\end{minipage}
\ifhmode\unskip\fi
\aftergroup\useignorespacesandallpars}

\def\useignorespacesandallpars#1\ignorespaces\fi{%
#1\fi\ignorespacesandallpars}

\makeatletter
\def\ignorespacesandallpars{%
  \@ifnextchar\par
    {\expandafter\ignorespacesandallpars\@gobble}%
    {}%
}
\makeatother


% definitions for citeproc citations
\NewDocumentCommand\citeproctext{}{}
\NewDocumentCommand\citeproc{mm}{%
\href{\#cite.\detokenize{#1}}{#2}\nocite{#1}}

\makeatletter
% allow citations to break across lines
\let\@cite@ofmt\@firstofone
% avoid brackets around text for \cite:
\def\@biblabel#1{}
\def\@cite#1#2{{#1\if@tempswa , #2\fi}}
\makeatother
\newlength{\cslhangindent}
\setlength{\cslhangindent}{1.5em}
\newlength{\csllabelwidth}
\setlength{\csllabelwidth}{3em}
\newenvironment{CSLReferences}[2] % #1 hanging-indent, #2 entry-spacing
{\begin{list}{}{%
	\setlength{\itemindent}{0pt}
	\setlength{\leftmargin}{0pt}
	\setlength{\parsep}{0pt}
	% turn on hanging indent if param 1 is 1
	\ifodd #1
	\setlength{\leftmargin}{\cslhangindent}
	\setlength{\itemindent}{-1\cslhangindent}
	\fi
	% set entry spacing
	\setlength{\itemsep}{#2\baselineskip}}}
{\end{list}}

\usepackage{calc}
\newcommand{\CSLBlock}[1]{\hfill\break\parbox[t]{\linewidth}{\strut\ignorespaces#1\strut}}
\newcommand{\CSLLeftMargin}[1]{\parbox[t]{\csllabelwidth}{\strut#1\strut}}
\newcommand{\CSLRightInline}[1]{\parbox[t]{\linewidth - \csllabelwidth}{\strut#1\strut}}
\newcommand{\CSLIndent}[1]{\hspace{\cslhangindent}#1}


\urlstyle{same}  % don't use monospace font for urls
\setlength{\parindent}{0pt}
\setlength{\parskip}{6pt plus 2pt minus 1pt}
\setlength{\emergencystretch}{3em}  % prevent overfull lines
\setcounter{secnumdepth}{5}

%%% Use protect on footnotes to avoid problems with footnotes in titles
\let\rmarkdownfootnote\footnote%
\def\footnote{\protect\rmarkdownfootnote}
\IfFileExists{upquote.sty}{\usepackage{upquote}}{}

%%% Include extra packages specified by user
\usepackage{sectsty}
\sectionfont{\bfseries\large}
\subsectionfont{\bfseries\normalsize}

%%% Hard setting column skips for reports - this ensures greater consistency and control over the length settings in the document.
%% page layout
%% paragraphs
\setlength{\baselineskip}{12pt plus 0pt minus 0pt}
\setlength{\parskip}{12pt plus 0pt minus 0pt}
\setlength{\parindent}{0pt plus 0pt minus 0pt}
%% floats
\setlength{\floatsep}{12pt plus 0 pt minus 0pt}
\setlength{\textfloatsep}{20pt plus 0pt minus 0pt}
\setlength{\intextsep}{14pt plus 0pt minus 0pt}
\setlength{\dbltextfloatsep}{20pt plus 0pt minus 0pt}
\setlength{\dblfloatsep}{14pt plus 0pt minus 0pt}
%% maths
\setlength{\abovedisplayskip}{12pt plus 0pt minus 0pt}
\setlength{\belowdisplayskip}{12pt plus 0pt minus 0pt}
%% lists
\setlength{\topsep}{10pt plus 0pt minus 0pt}
\setlength{\partopsep}{3pt plus 0pt minus 0pt}
\setlength{\itemsep}{5pt plus 0pt minus 0pt}
\setlength{\labelsep}{8mm plus 0mm minus 0mm}
\setlength{\parsep}{\the\parskip}
\setlength{\listparindent}{\the\parindent}
%% verbatim
\setlength{\fboxsep}{5pt plus 0pt minus 0pt}



\begin{document}



\begin{frontmatter}  %

\title{Macroeconomics Assignment}

% Set to FALSE if wanting to remove title (for submission)




\author[Add1]{Liam Andrew Beattie}
\ead{22562435@sun.ac.za}





\address[Add1]{Macroeconomics 871, Stellenbosch University, South
Africa}


\begin{abstract}
\small{
Abstract to be written here. The abstract should not be too long and
should provide the reader with a good understanding what you are writing
about. Academic papers are not like novels where you keep the reader in
suspense. To be effective in getting others to read your paper, be as
open and concise about your findings here as possible. Ideally, upon
reading your abstract, the reader should feel he / she must read your
paper in entirety.
}
\end{abstract}

\vspace{1cm}





\vspace{0.5cm}

\end{frontmatter}

\setcounter{footnote}{0}



%________________________
% Header and Footers
%%%%%%%%%%%%%%%%%%%%%%%%%%%%%%%%%
\pagestyle{fancy}
\chead{}
\rhead{}
\lfoot{}
\rfoot{\footnotesize Page \thepage}
\lhead{}
%\rfoot{\footnotesize Page \thepage } % "e.g. Page 2"
\cfoot{}

%\setlength\headheight{30pt}
%%%%%%%%%%%%%%%%%%%%%%%%%%%%%%%%%
%________________________

\headsep 35pt % So that header does not go over title




Start writing about what you are planning to do, note that this is an
assignment and what it is (introduction vibes)

Please give credit: I haved used an R package from Mati
(\citeproc{ref-Mati2019}{2019})

\section{Model Specification}\label{model-specification}

\#\# Core RBC Foundations

Above could note be a heading on itself. Just a paragraph that it is a
core rbc foundations and what that means.

Note that capital letters denote nominal amounts and lowercase denote
real values.

\subsection{Households}\label{households}

\begin{equation}
C_t \;=\; P_t \, c_t,
\qquad
I_t \;=\; P_t \, i_t
\label{nominal_definitions}
\end{equation}

\begin{equation}
\max_{\{C_t,\,h_t,\,M_t,\,B_t,\,K_t,\,I_t\}}
\mathbb{E}_0 \sum_{t=1}^{\infty} \beta^{t-1}
\left[
\frac{\Bigl(\frac{C_t}{P_t} - \eta\,\frac{C_{t-1}}{P_{t-1}}\Bigr)^{1-\theta}}{1-\theta}
\;-\;\chi\,\frac{h_t^{1+\gamma}}{1+\gamma}
\;+\;\psi\,\ln\!\Bigl(\frac{M_t}{P_t}\Bigr)
\right]
\label{lifetime_utility_nominal}
\end{equation}

\begin{equation}
C_t \;+\; I_t \;+\; B_t \;+\; M_t
\;\le\;
R^B_{\,t-1}\,B_{t-1}
\;+\; M_{t-1}
\;+\; W_t\,h_t
\;+\; R^k_t\,K_{t-1}
\;+\; \Pi_t
\;-\; P_t\,\tau_t
\label{flow_constraint_nominal}
\end{equation}

\begin{equation}
K_t
\;=\;
(1 - \delta)\,K_{t-1}
\;+\; I_t
\;-\;\frac{\phi}{2}
\left(\frac{I_t}{K_{t-1}} - \delta\right)^{2}
\,K_{t-1}
\label{capital_accumulation_nominal}
\end{equation}

\begin{center}\rule{0.5\linewidth}{0.5pt}\end{center}

From which we can derive the following equations, found in full form in
Appendix \ref{household_FOC}

\begin{equation}\label{consEuler}
  (c_t - \eta\,c_{t-1})^{-\theta}
  \;=\;
  \beta\,\mathbb{E}_t\!\Bigl[
    R_t^B \;\frac{P_t}{P_{t+1}}\;(c_{t+1} - \eta\,c_t)^{-\theta}
  \Bigr]
\end{equation}

Interpretation: Marginal rate of substitution between current and future
consumption equals the expected real bond return. Habit persistence
(\(\eta\)) links today's utility to past and future consumption, and
inflation (\(P_t/P_{t+1}\)) scales the real payoff on bonds.

\begin{equation}\label{labourSupply}
  \frac{W_t}{P_t}
  \;=\;
  \frac{\chi\,h_t^{\gamma}}
       {(c_t - \eta\,c_{t-1})^{-\theta}
        \;-\;
        \beta\,\eta\,\mathbb{E}_t\!\bigl[(c_{t+1} - \eta\,c_t)^{-\theta}\bigr]}
\end{equation}

Interpretation: Habit formation reduces effective marginal utility of
consumption, so stronger habits (\(\eta\uparrow\)) or more elastic
labour supply (\(\gamma\uparrow\)) require a higher real wage to induce
the same hours.

\begin{equation}\label{money_demand}
  M_t
  \;=\;
  \frac{\psi}
       {\beta\,\mathbb{E}_t\!\Bigl[
         (R_t^B - 1)
         \;\cdot\;
         \dfrac{(c_{t+1} - \eta\,c_t)^{-\theta}
               \;-\;
               \beta\,\eta\,\mathbb{E}_{t+1}[(c_{t+2} - \eta\,c_{t+1})^{-\theta}]}
              {P_{t+1}}
       \Bigr]}
\end{equation}

Interpretation: Higher expected nominal rates \((R_t^B - 1)\) raise the
opportunity cost of money, while habits and inflation expectations shape
the curvature of demand. The denominator captures the liquidity premium
adjusted for consumption dynamics.

\begin{equation}\label{capital_euler}
  \begin{gathered}
  q_t \;\equiv\; 1 - \phi\,\Bigl(\tfrac{I_t}{K_{t-1}} - \delta\Bigr) \\
  \\
  \frac{\beta\,\mathbb{E}_t\!\bigl[\lambda_{t+1}\,R_t^B\bigr]}{q_t}
  \;=\;
  \beta\,\mathbb{E}_t\!\Bigl[
    \lambda_{t+1}\Bigl(
      R_{t+1}^k
      + \frac{1}{q_{t+1}}
        \Bigl(
          1 - \delta
          + \tfrac{\phi}{2}\bigl[(I_{t+1}/K_t)^2 - \delta^2\bigr]
        \Bigr)
    \Bigr)
  \Bigr]
  \end{gathered}
\end{equation}

Interpretation: The bond-return-scaled discount factor divided by
\(q_t\) equals expected return on capital plus adjustment-cost terms. If
\(I_t/K_{t-1}>\delta\), then \(q_t>1\) signals profitable expansion;
disinvestment flips the sign. Adjustment costs (\(\phi\)) create
investment frictions.

\subsection{Production}\label{production}

While the RBC model features a single representative firm under perfect
competition, we now introduce two layers of firms to capture
monopolistic competition and nominal rigidities:\\
\textbf{1. Final goods producer} (perfectly competitive aggregator)\\
\textbf{2. Intermediate goods producers} (monopolistically competitive
with sticky prices)

\subsubsection{Final Goods Producer}\label{final-goods-producer}

\textbf{(Perfectly competitive, zero profits)}\\
\textbf{Role}: Aggregates differentiated inputs into final output.

\textbf{Key equations:}

\textbf{CES Production Function}:\\
\begin{equation}  
Y_t = \left( \int_0^1 Y_t(j)^{\frac{\epsilon-1}{\epsilon}}  dj \right)^{\frac{\epsilon}{\epsilon-1}}, \quad \epsilon > 1  
\label{ces_production}  
\end{equation}

\textbf{Demand for Good \(j\)}:\\
\begin{equation}  
Y_t(j) = \left( \frac{P_t(j)}{P_t} \right)^{-\epsilon} Y_t  
\label{demand_curve_final}  
\end{equation}

\textbf{Aggregate Price Index}:\\
\begin{equation}  
P_t = \left( \int_0^1 P_t(j)^{1-\epsilon}  dj \right)^{\frac{1}{1-\epsilon}}  
\label{aggregate_price_index}  
\end{equation}

Equations \ref{demand_curve_final} and \ref{aggregate_price_index} are
derived from \ref{ces_production} in Appendix
\ref{final_good_producer_appendix}.

\textbf{Economic Intuition:}

\begin{itemize}
\tightlist
\item
  \(\epsilon\): Elasticity of substitution (lower \(\epsilon\) implies
  more market power)
\item
  Demand for \(j\) decreases with its relative price
  \(\frac{P_t(j)}{P_t}\)
\end{itemize}

\subsubsection{Intermediate Goods
Producers}\label{intermediate-goods-producers}

\textbf{(Monopolistic competitors, indexed by \(j \in [0,1]\))}

\subsubsection{Technology and Factor
Inputs}\label{technology-and-factor-inputs}

\textbf{Production Function} (same as RBC): \[
Y_t(j) = A_t K_t(j)^\alpha h_t(j)^{1-\alpha}
\]

\textbf{Assumptions:}

\begin{itemize}
\tightlist
\item
  Identical technology \(A_t\) (common TFP shock)
\item
  Firms rent capital \(K_t(j)\) and labor \(h_t(j)\) from households
\item
  Competitive factor markets: prices \(R_t^k\), \(W_t\)
\end{itemize}

\subsubsection{Price-Setting Friction}\label{price-setting-friction}

\textbf{Rotemberg Price Adjustment Cost}: \[
\text{AdjCost}_t(j) = \frac{\psi}{2} \left( \frac{P_t(j)}{P_{t-1}(j)} - 1 \right)^2 Y_t
\]

\textbf{Intuition:}

\begin{itemize}
\tightlist
\item
  Convex cost for price changes (relative to past price)
\item
  \(\psi\): Adjustment cost parameter (higher → more stickiness)
\item
  Scaling by \(Y_t\) ensures cost grows with output
\end{itemize}

\begin{center}\rule{0.5\linewidth}{0.5pt}\end{center}

\subsection{Firm Optimization and
Equilibrium}\label{firm-optimization-and-equilibrium}

\subsubsection{Profit Maximization
Problem}\label{profit-maximization-problem}

\textbf{Objective}: Choose \(P_t(j)\) to maximize discounted real
profits: \[
\max_{P_t(j)} \mathbb{E}_t \sum_{s=0}^\infty \beta^s \frac{\lambda_{t+s}}{\lambda_t} \left[
\underbrace{\frac{P_t(j)}{P_{t+s}} Y_{t+s}(j)}_{\text{Revenue}} - 
\underbrace{\text{MC}_{t+s} Y_{t+s}(j)}_{\text{Cost}} - 
\underbrace{\text{AdjCost}_{t+s}(j)}_{\text{Price adjustment cost}}
\right]
\]

\textbf{Where:}

\begin{itemize}
\tightlist
\item
  \(\beta^s \frac{\lambda_{t+s}}{\lambda_t}\): Household stochastic
  discount factor
\item
  \(\text{MC}_t\): Real marginal cost (defined next)
\end{itemize}

\subsubsection{Key Equilibrium Concepts}\label{key-equilibrium-concepts}

\textbf{Marginal Cost}: \[
\text{MC}_t = \frac{1}{A_t} \left( \frac{R_t^k}{\alpha} \right)^\alpha \left( \frac{W_t}{1 - \alpha} \right)^{1 - \alpha}
\]

\textbf{Intuition}: Inverse of TFP times Cobb-Douglas cost

\textbf{Symmetric Equilibrium}: \textgreater{} All firms face identical
conditions, so \(P_t(j) = P_t\), \(Y_t(j) = Y_t\).

\subsubsection{New Keynesian Phillips Curve
(NKPC)}\label{new-keynesian-phillips-curve-nkpc}

\textbf{Core NKPC Equation}: \[
\psi (\Pi_t - 1) \Pi_t = \epsilon \left( 1 - \frac{\text{MC}_t}{\mu} \right) + 
\beta \psi \mathbb{E}_t \left[ \frac{\lambda_{t+1}}{\lambda_t} (\Pi_{t+1} - 1) \Pi_{t+1} \frac{Y_{t+1}}{Y_t} \right]
\]

\textbf{Definitions:}

\begin{itemize}
\tightlist
\item
  \(\Pi_t \equiv \frac{P_t}{P_{t-1}}\): Gross inflation
\item
  \(\mu \equiv \frac{\epsilon}{\epsilon - 1}\): Desired markup
\end{itemize}

\textbf{Interpretation:}

\begin{itemize}
\tightlist
\item
  LHS: Current inflation cost\\
\item
  RHS:

  \begin{itemize}
  \tightlist
  \item
    \(\epsilon (1 - \text{MC}_t / \mu)\): Markup gap\\
  \item
    Expectation term: Forward-looking inflation pressure
  \end{itemize}
\end{itemize}

\subsection{Resource Constraint}\label{resource-constraint}

\textbf{Final Output Allocation}: \[
Y_t = C_t + I_t + \frac{\psi}{2} (\Pi_t - 1)^2 Y_t
\]

\textbf{Interpretation}: Price adjustment costs are a deadweight loss.

Here's a concise explanation of how the firm section extends a simple
RBC model, integrating monopolistic competition and Rotemberg price
rigidity.

\subsection{From RBC to New Keynesian
Framework}\label{from-rbc-to-new-keynesian-framework}

In a basic RBC model, you have a single representative firm with perfect
competition and flexible prices. We extend this in two key ways:

\begin{enumerate}
\def\labelenumi{\arabic{enumi}.}
\item
  \textbf{Monopolistic Competition}\\
  Instead of one firm, we now have:

  \begin{itemize}
  \tightlist
  \item
    A \emph{final goods producer} that aggregates differentiated
    intermediate goods (from firms \(j \in [0,1]\)) into final output
    \(Y_t\) using a CES technology. This creates downward-sloping demand
    curves for each firm \(j\): \[
    Y_t(j) = \left(\frac{P_t(j)}{P_t}\right)^{-\epsilon} Y_t
    \] where \(\epsilon > 1\) governs market power (higher \(\epsilon\)
    = more competition).
  \end{itemize}
\item
  \textbf{Intermediate Firms with Sticky Prices}\\
  Each firm \(j\) produces with Cobb-Douglas technology: \[
  Y_t(j) = A_t K_t(j)^\alpha h_t(j)^{1-\alpha}
  \] but faces two new frictions:

  \begin{itemize}
  \tightlist
  \item
    \textbf{Market Power}: They set \(P_t(j) > \text{MC}_t\), where \[
    \text{MC}_t = \frac{1}{A_t} \left(\frac{R_t^k}{\alpha}\right)^\alpha \left(\frac{W_t}{1-\alpha}\right)^{1-\alpha}
    \]
  \item
    \textbf{Rotemberg Price Rigidity}: Changing prices incurs a
    quadratic cost: \[
    \text{AdjCost}_t = \frac{\psi}{2} \left(\frac{P_t(j)}{P_{t-1}(j)} - 1\right)^2 Y_t
    \] This penalizes large price changes (\(\psi\) controls
    stickiness).
  \end{itemize}
\end{enumerate}

\subsection{Price Setting Dynamics}\label{price-setting-dynamics}

Firms maximize \textbf{discounted real profits}: \[
\max_{P_t(j)} \mathbb{E}_t \sum_{s=0}^\infty \beta^s \frac{\lambda_{t+s}}{\lambda_t} \left[ 
\frac{P_t(j)}{P_{t+s}} Y_{t+s}(j) - \text{MC}_{t+s} Y_{t+s}(j) - \text{AdjCost}_{t+s} 
\right]
\]

\begin{itemize}
\item
  \(\beta^s \frac{\lambda_{t+s}}{\lambda_t}\): Household's stochastic
  discount factor (from your Euler equation).
\item
  In symmetric equilibrium (\(P_t(j) = P_t\), \(Y_t(j) = Y_t\)), this
  yields the \textbf{New Keynesian Phillips Curve}: \[
  \psi (\Pi_t - 1) \Pi_t = \epsilon \left(1 - \frac{\text{MC}_t}{\mu} \right) + \beta \psi \mathbb{E}_t \left[ \frac{\lambda_{t+1}}{\lambda_t} (\Pi_{t+1} - 1) \Pi_{t+1} \frac{Y_{t+1}}{Y_t} \right]
  \] where \(\Pi_t \equiv \frac{P_t}{P_{t-1}}\) is inflation, and
  \(\mu \equiv \frac{\epsilon}{\epsilon - 1}\) is the markup.
\end{itemize}

This equation links inflation to marginal costs (real side) and expected
inflation (forward-looking term).

\subsection{Key Implications for RBC
Foundations}\label{key-implications-for-rbc-foundations}

\begin{itemize}
\item
  \textbf{Real Variables}: Production still uses \(K_{t-1}\) and \(h_t\)
  (as in RBC), but now inputs are rented from households at competitive
  rates \(R_t^k\) and \(W_t\).
\item
  \textbf{New Nominal Rigidity}: The adjustment cost
  \(\psi (\Pi_t - 1)^2 Y_t\) appears in the resource constraint, acting
  like a ``friction tax'' on output.
\item
  \textbf{Monetary Policy Transmission}: Interest rates (from your bond
  Euler equation) now affect real activity via inflation dynamics.
\end{itemize}

This structure preserves RBC foundations while adding nominal rigidities
and imperfect competition --- essential for analyzing monetary policy.
The only new state variable is \(P_{t-1}\) (for inflation dynamics),
maintaining tractability.

\newpage

\emph{Final Good Producer} The final good \(Y_t\) is produced by
combining intermediate goods \(Y_t(j)\):

\begin{equation}
Y_t = \left( \int_0^1 Y_t(j)^{\frac{\epsilon-1}{\epsilon}} dj \right)^{\frac{\epsilon}{\epsilon-1}}
\label{final_good_production}
\end{equation}

Profit maximization yields demand for each intermediate good:
\begin{equation}
Y_t(j) = \left( \frac{P_t(j)}{P_t} \right)^{-\epsilon} Y_t
\label{demand_intermediate}
\end{equation}

and the aggregate price index: \begin{equation}
P_t = \left( \int_0^1 P_t(j)^{1-\epsilon} dj \right)^{\frac{1}{1-\epsilon}}
\label{price_index}
\end{equation}

\emph{Intermediate Goods Producers} Each firm \(j\) produces with
Cobb-Douglas technology: \begin{equation}
Y_t(j) = A_t K_t(j)^{\alpha} h_t(j)^{1-\alpha}
\label{intermediate_production}
\end{equation}

Minimizing costs yields nominal marginal cost (common to all firms):
\begin{equation}
\text{MC}_t = \frac{1}{A_t} \left( \frac{R_t^k}{\alpha} \right)^{\alpha} \left( \frac{W_t}{1-\alpha} \right)^{1-\alpha}
\label{marginal_cost}
\end{equation}

Firms face Rotemberg (1982) price adjustment costs. The real profit
function is: \begin{equation}
\Pi_t(j) = \underbrace{\frac{P_t(j)}{P_t} Y_t(j)}_{\text{real revenue}} - \underbrace{\text{MC}_t \cdot Y_t(j)}_{\text{real cost}} - \underbrace{\frac{\psi}{2} \left( \frac{P_t(j)}{P_{t-1}(j)} - 1 \right)^2 Y_t}_{\text{adjustment cost}}
\label{firm_profit}
\end{equation}

\emph{Price Setting} Firms maximize discounted future profits:
\begin{equation}
\max_{P_t(j)} \mathbb{E}_t \sum_{s=0}^{\infty} \beta^s \frac{\lambda_{t+s}}{\lambda_t} \Pi_{t+s}(j)
\label{firm_objective}
\end{equation}

subject to demand (\ref{demand_intermediate}). In symmetric equilibrium
(\(P_t(j) = P_t\), \(Y_t(j) = Y_t\)), we obtain the Rotemberg Phillips
Curve: \begin{equation}
\psi (\Pi_t - 1) \Pi_t = \epsilon \left(1 - \frac{\text{MC}_t}{\mu}\right) + \beta \psi \mathbb{E}_t \left[ \frac{\lambda_{t+1}}{\lambda_t} (\Pi_{t+1} - 1) \Pi_{t+1} \frac{Y_{t+1}}{Y_t} \right]
\label{phillips_curve}
\end{equation} where \(\Pi_t \equiv P_t/P_{t-1}\) and
\(\mu \equiv \epsilon/(\epsilon-1)\).

\emph{Market Clearing} Aggregate production: \begin{equation}
Y_t = A_t K_{t-1}^{\alpha} h_t^{1-\alpha}
\label{aggregate_production}
\end{equation}

Resource constraint (adjustment costs reduce output): \begin{equation}
Y_t = C_t + I_t + \underbrace{\frac{\psi}{2} (\Pi_t - 1)^2 Y_t}_{\text{price adjustment cost}}
\label{resource_constraint}
\end{equation}

Factor markets clear: \begin{align}
\int_0^1 h_t(j) dj &= h_t \\
\int_0^1 K_t(j) dj &= K_{t-1}
\end{align}

\newpage
\newpage

\section{Appendix}\label{appendix}

\subsection{Households}\label{households-1}

\begin{align*}
  & \text{Define Lagrangian} \\
  & \mathcal{L} = \mathbb{E}_0 \sum_{t=1}^{\infty} \beta^{t-1} \Bigl\{
    \underbrace{\frac{\bigl(\tfrac{C_t}{P_t}-\eta\tfrac{C_{t-1}}{P_{t-1}}\bigr)^{1-\theta}}{1-\theta}}_{\text{Consumption utility}}
    - \underbrace{\chi\frac{h_t^{1+\gamma}}{1+\gamma}}_{\text{Labor disutility}}
    + \underbrace{\psi\ln\bigl(\tfrac{M_t}{P_t}\bigr)}_{\text{Money utility}} \\
  & \quad\qquad
    + \underbrace{\lambda_t\bigl[R^B_{t-1}B_{t-1}+M_{t-1}+W_th_t+R^k_tK_{t-1}+\Pi_t-P_t\tau_t-C_t-I_t-B_t-M_t\bigr]}_{\text{Nominal flow constraint}}
    + \underbrace{\mu_t\bigl[(1-\delta)K_{t-1}+I_t-\tfrac{\phi}{2}(\tfrac{I_t}{K_{t-1}}-\delta)^2K_{t-1}-K_t\bigr]}_{\text{Capital accumulation}}
  \Bigr\}
\end{align*}

\subsubsection{\texorpdfstring{First Order Conditions
\label{household_FOC}}{First Order Conditions }}\label{first-order-conditions}

\textbf{FOC w.r.t. Consumption} : \begin{align*}
  & \frac{\partial \mathcal{L}}{\partial C_t} = 0 \\
  & \quad \bigl[(c_t-\eta c_{t-1})^{-\theta}/P_t - \lambda_t\bigr]
    - \beta\,\mathbb{E}_t\bigl[\eta\,(c_{t+1}-\eta c_t)^{-\theta}/P_t\bigr] = 0 \\[6pt]
  & \text{Combine terms over }1/P_t \\
  & \quad \frac{1}{P_t}\bigl[(c_t-\eta c_{t-1})^{-\theta} - \beta\eta\,\mathbb{E}_t[(c_{t+1}-\eta c_t)^{-\theta}]\bigr] - \lambda_t = 0 \\[6pt]
  & \text{Multiply by }P_t \\
  & \quad (c_t-\eta c_{t-1})^{-\theta} - \beta\eta\,\mathbb{E}_t[(c_{t+1}-\eta c_t)^{-\theta}] - \lambda_t P_t = 0
\end{align*}

\begin{equation}\label{foc_C}
\boxed{
  \lambda_t P_t = (c_t-\eta c_{t-1})^{-\theta} - \beta\eta\,\mathbb{E}_t\bigl[(c_{t+1}-\eta c_t)^{-\theta}\bigr]
}
\end{equation}

\textbf{FOC w.r.t. Labour} : \begin{align*}
  & \frac{\partial \mathcal{L}}{\partial h_t} = 0 \\
  & \quad -\chi h_t^{\gamma} + \lambda_t W_t = 0 \\[6pt]
  & \text{Rearrange} \\
  & \quad \lambda_t W_t = \chi h_t^{\gamma}
\end{align*}

\begin{equation}\label{foc_h}
\boxed{\lambda_t W_t = \chi h_t^{\gamma}}
\end{equation}

\textbf{FOC w.r.t. Real Money Balances}: \begin{align*}
  & \frac{\partial \mathcal{L}}{\partial M_t} = 0 \\
  & \quad \beta^{t-1}\bigl[\psi/M_t - \lambda_t\bigr] + \beta^t\mathbb{E}_t[\lambda_{t+1}] = 0 \\[6pt]
  & \text{Divide by }\beta^{t-1}\text{ and rearrange} \\
  & \quad \psi/M_t - \lambda_t + \beta\,\mathbb{E}_t[\lambda_{t+1}] = 0
\end{align*}

\begin{equation}\label{foc_M}
\boxed{\frac{\psi}{M_t} = \lambda_t - \beta\,\mathbb{E}_t[\lambda_{t+1}]}
\end{equation}

\textbf{FOC w.r.t. Bonds} \eqref{foc_B}): \begin{align*}
  & \frac{\partial \mathcal{L}}{\partial B_t} = 0 \\
  & \quad -\beta^{t-1}\lambda_t + \beta^t\mathbb{E}_t[\lambda_{t+1}R^B_t] = 0 \\[6pt]
  & \text{Divide by }\beta^{t-1}\text{ and simplify} \\
  & \quad -\lambda_t + \beta\,\mathbb{E}_t[\lambda_{t+1}R^B_t] = 0
\end{align*}

\begin{equation}\label{foc_B}
\boxed{\lambda_t = \beta\,\mathbb{E}_t[\lambda_{t+1}R^B_t]}
\end{equation}

\textbf{FOC w.r.t. Capital} : \begin{align*}
  & \frac{\partial \mathcal{L}}{\partial K_t} = 0 \\
  & \quad -\beta^{t-1}\mu_t + \beta^t\mathbb{E}_t\bigl[\lambda_{t+1}R^k_{t+1} + \mu_{t+1}(1-\delta + \tfrac{\phi}{2}((I_{t+1}/K_t)^2 - \delta^2))\bigr] = 0 \\[6pt]
  & \text{Divide by }\beta^{t-1}\text{ and solve} \\
  & \quad \mu_t = \beta\,\mathbb{E}_t\bigl[\lambda_{t+1}R^k_{t+1} + \mu_{t+1}(1-\delta + \tfrac{\phi}{2}((I_{t+1}/K_t)^2 - \delta^2))\bigr]
\end{align*}

\begin{equation}\label{foc_K}
\boxed{\mu_t = \beta\,\mathbb{E}_t\bigl[\lambda_{t+1}R^k_{t+1} + \mu_{t+1}(1-\delta + \tfrac{\phi}{2}((I_{t+1}/K_t)^2 - \delta^2))\bigr]}
\end{equation}

\textbf{FOC w.r.t. Investment} : \begin{align*}
  & \frac{\partial \mathcal{L}}{\partial I_t} = 0 \\
  & \quad \beta^{t-1}\bigl[-\lambda_t + \mu_t(1 - \phi(\tfrac{I_t}{K_{t-1}} - \delta))\bigr] = 0 \\[6pt]
  & \text{Divide by }\beta^{t-1}\text{ and isolate} \\
  & \quad \lambda_t = \mu_t\bigl(1 - \phi(\tfrac{I_t}{K_{t-1}} - \delta)\bigr)
\end{align*}

\begin{equation}\label{foc_I}
\boxed{\lambda_t = \mu_t\bigl(1 - \phi(\tfrac{I_t}{K_{t-1}} - \delta)\bigr)}
\end{equation}

\subsubsection{Household Final
Equations}\label{household-final-equations}

\textbf{Consumption Euler Equation}\\
Combines consumption--habit dynamics with bond returns (from
\eqref{foc_B} and \eqref{foc_C}) :

\begin{align*}
& \text{Start with FOC for Bonds} \\
& \lambda_t = \beta\,\mathbb{E}_t[\lambda_{t+1}R_t^B] \quad \text{(Equation \ref{foc_B})} \\
& \\
& \text{Substitute } \lambda_t \text{ and } \lambda_{t+1} \text{ from FOC for Consumption} \\
& \lambda_t = \frac{(c_t - \eta\,c_{t-1})^{-\theta}}{P_t} \quad \text{(from Equation \ref{foc_C} rearranged)} \\
& \lambda_{t+1} = \frac{(c_{t+1} - \eta\,c_t)^{-\theta}}{P_{t+1}} \quad \text{(time-shifted)} \\
& \\
& text{Combine results} \\
& \frac{(c_t - \eta\,c_{t-1})^{-\theta}}{P_t} = \beta\,\mathbb{E}_t\!\left[ R_t^B \cdot \frac{(c_{t+1} - \eta\,c_t)^{-\theta}}{P_{t+1}} \right] \\
& \\
& \text{Clear denominator} \\
& (c_t - \eta\,c_{t-1})^{-\theta} = \beta\,\mathbb{E}_t\!\left[ R_t^B \cdot \frac{P_t}{P_{t+1}} \cdot (c_{t+1} - \eta\,c_t)^{-\theta} \right]
\end{align*}

\begin{equation}\label{consEuler_app}
\boxed{%
  (c_t - \eta\,c_{t-1})^{-\theta}
  \;=\;
  \beta\,\mathbb{E}_t\!\Bigl[
    R_t^B \;\frac{P_t}{P_{t+1}}\;(c_{t+1} - \eta\,c_t)^{-\theta}
  \Bigr]
}
\end{equation}

Interpretation: Marginal rate of substitution between current and future
consumption equals the expected real bond return. Habit persistence
(\(\eta\)) links today's utility to past and future consumption, and
inflation (\(P_t/P_{t+1}\)) scales the real payoff on bonds.

\textbf{Labour Supply}

Real wage equals the marginal rate of substitution between leisure and
consumption (from \eqref{foc_C} and \eqref{foc_h}) :

\begin{align*}
& \text{Start with FOC for Hours Worked} \\
& \lambda_t W_t = \chi\,h_t^{\gamma} \quad \text{(Equation \ref{foc_h})} \\
& \\
& \text{Solve for } \lambda_t \\
& \lambda_t = \frac{\chi\,h_t^{\gamma}}{W_t} \\
& \\
& \text{Equate to FOC of Consumption expression} \\
& \frac{\chi\,h_t^{\gamma}}{W_t} = \frac{(c_t - \eta\,c_{t-1})^{-\theta} - \beta\,\eta\,\mathbb{E}_t[(c_{t+1} - \eta\,c_t)^{-\theta}]}{P_t} \\
& \\
& \text{Solve for real wage } (W_t/P_t) \\
& \frac{W_t}{P_t} = \frac{\chi\,h_t^{\gamma}}{(c_t - \eta\,c_{t-1})^{-\theta} - \beta\,\eta\,\mathbb{E}_t[(c_{t+1} - \eta\,c_t)^{-\theta}]}
\end{align*}

\begin{equation}\label{labourSupply_app}
\boxed{
  \frac{W_t}{P_t}
  \;=\;
  \frac{\chi\,h_t^{\gamma}}
       {(c_t - \eta\,c_{t-1})^{-\theta}
        \;-\;
        \beta\,\eta\,\mathbb{E}_t\!\bigl[(c_{t+1} - \eta\,c_t)^{-\theta}\bigr]}
}
\end{equation}

Interpretation: Habit formation reduces effective marginal utility of
consumption, so stronger habits (\(\eta\uparrow\)) or more elastic
labour supply (\(\gamma\uparrow\)) require a higher real wage to induce
the same hours.

\textbf{Money Demand}\\
Opportunity cost of holding money vs.~bonds (from \eqref{foc_M},
\eqref{foc_B} and \eqref{foc_C}) :

\begin{align*}
& \text{Combine FOC for Money and Bonds} \\
& \frac{\psi}{M_t} = \lambda_t - \beta\,\mathbb{E}_t[\lambda_{t+1}] \quad \text{(Equation \ref{foc_M})} \\
& \lambda_t = \beta\,\mathbb{E}_t[\lambda_{t+1}R_t^B] \quad \text{(Equation \ref{foc_B})} \\
& \\
& \text{Substitute } \lambda_t \text{ into FOC of money} \\
& \frac{\psi}{M_t} = \beta\,\mathbb{E}_t[\lambda_{t+1}R_t^B] - \beta\,\mathbb{E}_t[\lambda_{t+1}] \\
& \frac{\psi}{M_t} = \beta\,\mathbb{E}_t\left[\lambda_{t+1}(R_t^B - 1)\right] \\
& \\
& \text{Substitute } \lambda_{t+1} \text{ from FOC of Consumption} \\
& \lambda_{t+1} = \frac{(c_{t+1} - \eta\,c_t)^{-\theta} - \beta\,\eta\,\mathbb{E}_{t+1}[(c_{t+2} - \eta\,c_{t+1})^{-\theta}]}{P_{t+1}} \\
& \\
& \text{Solve for } M_t \\
& M_t = \frac{\psi}{\beta\,\mathbb{E}_t\!\left[ (R_t^B - 1) \cdot \dfrac{(c_{t+1} - \eta\,c_t)^{-\theta} - \beta\,\eta\,\mathbb{E}_{t+1}[(c_{t+2} - \eta\,c_{t+1})^{-\theta}]}{P_{t+1}} \right]}
\end{align*}

\begin{equation}\label{money_demand_app}
\boxed{
  M_t
  \;=\;
  \frac{\psi}
       {\beta\,\mathbb{E}_t\!\Bigl[
         (R_t^B - 1)
         \;\cdot\;
         \dfrac{(c_{t+1} - \eta\,c_t)^{-\theta}
               \;-\;
               \beta\,\eta\,\mathbb{E}_{t+1}[(c_{t+2} - \eta\,c_{t+1})^{-\theta}]}
              {P_{t+1}}
       \Bigr]}
}
\end{equation} Interpretation: Higher expected nominal rates
\((R_t^B - 1)\) raise the opportunity cost of money, while habits and
inflation expectations shape the curvature of demand. The denominator
captures the liquidity premium adjusted for consumption dynamics.

\textbf{Capital Euler Equation } Defines Tobin's \(q\) and links
required returns on capital to bond returns (from \eqref{foc_I},
\eqref{foc_K} and \eqref{foc_B}) :

\begin{align*}
& \text{Define Tobin's } q \text{ from FOC for Investment} \\
& \lambda_t = \mu_t q_t \quad \text{where} \quad q_t \equiv 1 - \phi\left(\tfrac{I_t}{K_{t-1}} - \delta\right) \\
& \\
& \text{Rearrange FOC for Capital} \\
& \mu_t = \beta\,\mathbb{E}_t\!\left[ \lambda_{t+1}R_{t+1}^k + \mu_{t+1}\!\left(1-\delta + \tfrac{\phi}{2}\left[(I_{t+1}/K_t)^2 - \delta^2\right]\right) \right] \\
& \\
& \text{Substitute } \mu_t = \lambda_t / q_t \text{ and } \mu_{t+1} = \lambda_{t+1} / q_{t+1} \\
& \frac{\lambda_t}{q_t} = \beta\,\mathbb{E}_t\!\left[ \lambda_{t+1}R_{t+1}^k + \frac{\lambda_{t+1}}{q_{t+1}}\left(1-\delta + \tfrac{\phi}{2}\left[(I_{t+1}/K_t)^2 - \delta^2\right]\right) \right] \\
& \\
& \text{Factor } \lambda_{t+1} \\
& \frac{\lambda_t}{q_t} = \beta\,\mathbb{E}_t\!\left[ \lambda_{t+1} \left( R_{t+1}^k + \frac{1}{q_{t+1}}\left(1-\delta + \tfrac{\phi}{2}\left[(I_{t+1}/K_t)^2 - \delta^2\right]\right) \right) \right] \\
& \\
& \text{Substitute FOC for Bonds } (\lambda_t = \beta\,\mathbb{E}_t[\lambda_{t+1}R_t^B]) \\
& \frac{\beta\,\mathbb{E}_t[\lambda_{t+1}R_t^B]}{q_t} = \beta\,\mathbb{E}_t\!\left[ \lambda_{t+1} \left( R_{t+1}^k + \frac{1}{q_{t+1}}\Gamma_{t+1} \right) \right] \\
& \text{where } \Gamma_{t+1} \equiv 1-\delta + \tfrac{\phi}{2}\left[(I_{t+1}/K_t)^2 - \delta^2\right]
\end{align*}

\begin{equation}\label{capital_euler_app}
\boxed{
  \begin{gathered}
  q_t \;\equiv\; 1 - \phi\,\Bigl(\tfrac{I_t}{K_{t-1}} - \delta\Bigr) \\
  \\
  \frac{\beta\,\mathbb{E}_t\!\bigl[\lambda_{t+1}\,R_t^B\bigr]}{q_t}
  \;=\;
  \beta\,\mathbb{E}_t\!\Bigl[
    \lambda_{t+1}\Bigl(
      R_{t+1}^k
      + \frac{1}{q_{t+1}}
        \Bigl(
          1 - \delta
          + \tfrac{\phi}{2}\bigl[(I_{t+1}/K_t)^2 - \delta^2\bigr]
        \Bigr)
    \Bigr)
  \Bigr]
  \end{gathered}
}
\end{equation}

Interpretation: The bond-return-scaled discount factor divided by
\(q_t\) equals expected return on capital plus adjustment-cost terms. If
\(I_t/K_{t-1}>\delta\), then \(q_t>1\) signals profitable expansion;
disinvestment flips the sign. Adjustment costs (\(\phi\)) create
investment frictions.

\subsection{Production}\label{production-1}

\subsubsection{\texorpdfstring{Final Good Producer
\label{final_good_producer_appendix}}{Final Good Producer }}\label{final-good-producer}

\textbf{Derivation of Intermediate Goods Demand and Aggregate Price
Index}

\begin{align*}  
& \text{Final goods producer's profit:} \\  
& \Pi_t = P_t Y_t - \int_0^1 P_t(j) Y_t(j)  dj \\  
& \text{subject to } Y_t = \left( \int_0^1 Y_t(j)^{\frac{\epsilon-1}{\epsilon}}  dj \right)^{\frac{\epsilon}{\epsilon-1}} \\  
& \\  
& \text{Substitute production function into profit:} \\  
& \Pi_t = P_t \left( \int_0^1 Y_t(j)^{\frac{\epsilon-1}{\epsilon}}  dj \right)^{\frac{\epsilon}{\epsilon-1}} - \int_0^1 P_t(j) Y_t(j)  dj \\  
& \\  
& \text{First-order condition for } Y_t(j): \\  
& \frac{\partial \Pi_t}{\partial Y_t(j)} = P_t \cdot \frac{\epsilon}{\epsilon-1} \left( \int_0^1 Y_t(i)^{\frac{\epsilon-1}{\epsilon}}  di \right)^{\frac{1}{\epsilon-1}} \cdot \frac{\epsilon-1}{\epsilon} Y_t(j)^{-\frac{1}{\epsilon}} - P_t(j) = 0 \\  
& \Rightarrow P_t \cdot Y_t^{\frac{1}{\epsilon}} Y_t(j)^{-\frac{1}{\epsilon}} = P_t(j) \\  
& \\  
& \text{Rearrange to obtain demand curve:} \\  
& Y_t(j) = \left( \frac{P_t}{P_t(j)} \right)^{\epsilon} Y_t \\  
& \\  
& \text{Substitute demand into production function:} \\  
& Y_t = \left( \int_0^1 \left[ \left( \frac{P_t}{P_t(j)} \right)^{\epsilon} Y_t \right]^{\frac{\epsilon-1}{\epsilon}}  dj \right)^{\frac{\epsilon}{\epsilon-1}} \\  
& = Y_t \left( \int_0^1 \left( \frac{P_t}{P_t(j)} \right)^{\epsilon-1}  dj \right)^{\frac{\epsilon}{\epsilon-1}} \\  
& \\
\end{align*} \begin{align*} 
& \text{Simplify to obtain price index:} \\  
& 1 = \left( \int_0^1 \left( \frac{P_t}{P_t(j)} \right)^{\epsilon-1}  dj \right)^{\frac{\epsilon}{\epsilon-1}} \\  
& \Rightarrow P_t^{1-\epsilon} = \int_0^1 P_t(j)^{1-\epsilon}  dj \\  
& \Rightarrow P_t = \left( \int_0^1 P_t(j)^{1-\epsilon}  dj \right)^{\frac{1}{1-\epsilon}}  
\end{align*}

\begin{equation}\label{demand_and_price}  
\boxed{  
  \begin{gathered}  
  Y_t(j) = \left( \frac{P_t(j)}{P_t} \right)^{-\epsilon} Y_t \\  
  \\  
  P_t = \left( \int_0^1 P_t(j)^{1-\epsilon}  dj \right)^{\frac{1}{1-\epsilon}}  
  \end{gathered}  
}  
\end{equation}

\newpage

\newpage

\section{Old Stuff}\label{old-stuff}

Components breakdown:

Expenditures:

Consumption: \(P_t c_t\)

Investment: \(P_t i_t\)

Bonds: \(B_t\)

Money holdings: \(M_t\)

Income sources:

Bond returns: \((1 + i_{t-1}) B_{t-1}\)

Money carryover: \(M_{t-1}\)

Labor income: \(W_t h_t\)

Capital returns: \(R_t^k K_{t-1}\) (Key addition missing in Sims)

Firm profits: \(\Pi_t\)

Net transfers: \(-P_t \tau_t\)

The utility function above (\ref{household_utility_fun}) must be
maximised subject to some sort of flow constraint. Note that the flow
budget is undefined as of now because i am unsure if captial shows up
there (which it should), bonds must too, hours worked and consumption
(taxes too surely?)

\begin{equation}
K_t = (1 - \delta) K_{t-1} + i_t - \frac{\phi}{2} \left( \frac{i_t}{K_{t-1}} - \delta \right)^2 K_{t-1}
\end{equation}

Key Improvements over Sims Capital integration:

Explicit rental rate \(R_t^k\) for capital services

Physical capital stock \(K_t\) in accumulation process

Convex adjustment costs (\(\phi > 0\))

Real money balances:

Maintains money-in-utility (MIU) specification

Consistent with Walsh (2010) framework

Habit persistence:

\(c_t - \eta c_{t-1}\) with \(\eta \in (0,1)\)

Generates consumption inertia matching SA data

\subsection{Firm Sector with Nominal
Rigidities}\label{firm-sector-with-nominal-rigidities}

Production function with capital:

\begin{equation}
Y_t(i) = A_t K_t(i)^{\alpha} H_t(i)^{1-\alpha}, \quad \alpha \in (0,1)
\label{production}
\end{equation}

Cost minisations:

\begin{equation}
\min_{K_t(i), H_t(i)} R_t^k K_t(i) + W_t H_t(i) \quad \text{s.t.} \quad Y_t(i) = A_t K_t(i)^{\alpha} H_t(i)^{1-\alpha}
\end{equation}

\subsubsection{Perfect competion final goods
firl}\label{perfect-competion-final-goods-firl}

CES aggreagation:

\begin{equation}
Y_t = \left( \int_0^1 Y_t(i)^{\frac{\epsilon-1}{\epsilon}} di \right)^{\frac{\epsilon}{\epsilon-1}}, \quad \epsilon > 1
\label{ces_aggregator}
\end{equation}

demand function:

\begin{equation}
Y_t(i) = \left( \frac{P_t(i)}{P_t} \right)^{-\epsilon} Y_t
\label{demand_curve}
\end{equation}

\subsubsection{Pricing setting (calvo)}\label{pricing-setting-calvo}

\begin{equation}
P_t^* = \frac{\epsilon}{\epsilon-1} \frac{
\mathbb{E}_t \sum_{k=0}^{\infty} (\beta\theta)^k \lambda_{t+k} MC_{t+k} P_{t+k}^{\epsilon} Y_{t+k}
}{
\mathbb{E}_t \sum_{k=0}^{\infty} (\beta\theta)^k \lambda_{t+k} P_{t+k}^{\epsilon-1} Y_{t+k}
}
\label{optimal_price}
\end{equation}

Price index dynamics:

\begin{equation}
P_t^{1-\epsilon} = \theta P_{t-1}^{1-\epsilon} + (1-\theta)(P_t^*)^{1-\epsilon}
\label{price_index}
\end{equation}

devidenet distribution:

\begin{equation}
\Pi_t = \int_0^1 \left[ P_t(i)Y_t(i) - W_t H_t(i) - R_t^k K_t(i) \right] di
\label{dividends}
\end{equation}

Aggregate equivalent:

\begin{equation}
\Pi_t = P_t Y_t - W_t H_t - R_t^k K_t
\label{agg_dividends}
\end{equation}

\(Y_t = A_t K_t^\alpha H_t^{1-\alpha}\) Cobb-Douglas

Capital accumulation:
\(K_{t+1} = (1-\delta)K_t + I_t - \frac{\phi}{2}\left(\frac{I_t}{K_t} - \delta\right)^2 K_t\)

\subsubsection{Nominal Rigidities}\label{nominal-rigidities}

Probability \(\theta=0.75\) of price non-adjustment

Phillips Curve derivation: \(\pi_t = \beta E_t\pi_{t+1} + \kappa mc_t\)

Dividend specification: \(\Pi_t = Y_t - w_t h_t - r_t^k k_t\)

\subsection{Government Sector}\label{government-sector}

Fiscal rule: \(T_t = \tau Y_t\) (lump-sum taxes)

Monetary authority: - Taylor Rule:
\(R_t = \rho R_{t-1} + (1-\rho)[\phi_\pi \pi_t + \phi_y \hat{Y}_t] + \varepsilon_t^r\)

-Money Growth Rule:
\(\ln \mu_t = \rho_\mu \ln \mu_{t-1} + \varepsilon_t^m\)

\subsection{Exogenous Processes}\label{exogenous-processes}

\begin{itemize}
\item
  TFP shock:
  \(\ln A_t = (1-\rho_A)\ln A_{ss} + \rho_A \ln A_{t-1} + \varepsilon_t^A\)
\item
  Monetary policy shocks (\(\varepsilon_t^r, \varepsilon_t^m\)
\end{itemize}

\subsection{Equilibrium and Model
Closure}\label{equilibrium-and-model-closure}

\begin{itemize}
\item
  Output Gap: \(\hat{Y}_t = Y_t - Y_t^n\) (natural rate output)
\item
  Market clearing conditions
\item
  Determinacy Requirements: Blanchard-Kahn conditions for policy rules
\end{itemize}

\section{Solution Strategy}\label{solution-strategy}

\subsection{Steady State Derivation}\label{steady-state-derivation}

\subsection{Log-Linearization
Techniques}\label{log-linearization-techniques}

\subsection{Determinacy Analysis}\label{determinacy-analysis}

\section{The Steady State}\label{the-steady-state}

Taylor principle verification (\(\phi_\pi > 1\))

\begin{figure}
\centering
\includegraphics{code/rbc_model/rbc_model/graphs/rbc_model_IRF_eps_cropped.png}
\caption{image}
\end{figure}

\section{Parameterization}\label{parameterization}

\subsection{Calibration Table}\label{calibration-table}

\subsection{Data Alignment}\label{data-alignment}

\section{Quantitative Analysis}\label{quantitative-analysis}

\begin{verbatim}
## [1] "works"
\end{verbatim}

\newpage

\section*{References}\label{references}
\addcontentsline{toc}{section}{References}

\phantomsection\label{refs}
\begin{CSLReferences}{1}{1}
\bibitem[\citeproctext]{ref-Mati2019}
Mati, S. 2019. DynareR: Bringing the power of {Dynare} to {R}, {R
Markdown}, and {Quarto}. \emph{CRAN}. {[}Online{]}, Available:
\url{https://CRAN.R-project.org/package=DynareR}.

\end{CSLReferences}

\bibliography{Tex/ref}





\end{document}
